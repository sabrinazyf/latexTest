\documentclass[a4]{jarticle}
%\usepackage{html,htmllist}
%\usepackage{lgrind}
%\usepackage[dvips]{graphicx,color}
\usepackage[dvipdfmx]{graphicx,color}
\usepackage{amsmath}
\usepackage{amsthm}

\topmargin=-2.0cm
\textheight=25.0cm
\textwidth=17.0cm
\oddsidemargin=-1.0cm
\evensidemargin=-1.0cm
\renewcommand{\baselinestretch}{0.95}%ページ数が伸びたときにはここに
%1.0以下の数字を入れてみましょう

\title{Operating Instruction of Zheng Yufei}
%\author{B4 鄭 中翔} %名前、学年を入れましょう
\author{鄭 雨霏} %名前、学年を入れましょう

\begin{document}
    \twocolumn{
    \maketitle
    }

    \section{Factory Settings}


    1 Factory Settings
    Zheng Yufei, born in April 18th, 1996, female. She graduated from
    Sun Yat-sen University in 2014. Her undergraduate major is Mathematics
    and Applied Mathematics. She has very broad interests but none of
    them is professional. Although she is a quick learner and good at
    asking questions, she likes to procrastinate her work. If anyone finds
    her absent-minded, please remind her of the work must to be done and
    tell her the dead line is very close. Her built-in programs don’t
    include human-face-recognition or map navigation. As a result, she
    often gets lost and suffer from face blindness. All in all, Zheng Yufei
    is very easy to get along with, don’t worry.


    \section{Personal Interests and Preferences}


    2 Personal Interests and Preferences
    Zheng Yufei has many showy and not substantial skills, including singing,
    playing the piano, playing the violin, drawing, playing various courtgames
    and so on. After many years of coding, most of them have been forgotten
    successfully. If you're interested, you can talk to her to find more information.
    For the moment, she likes playing video games and watching Japanese animations.

    \section{印22刷}
    両面印刷を心がけて下さい。両面印刷では長辺で折り返すようにします。
    2頁以上の文書は縮小して、2ページがA4一枚に入るようにしてください。
    コピーには5Fにあるコピー機が使えます。コピーカードは研究室に2枚だけ(白
    黒限定の一枚と白黒カラー可能な一枚)
    渡されていますので、なくさないようにしましょう。

    \section{論文の章構成}

    資料は最終的にまとめて論文の形にします。論文には学会へ投稿するもの、大
    学を修了するために提出するものがあります。どちらも
    章構成はほぼ定形です。基本形は以下のとおりですので、従いましょう。この
    順序は卒論、修論、博論すべて共通です。研究相談資料は論文を書く時に役立
    つように丁寧に書きましょう。

    \begin{enumerate}
        \item 概要(abstract)
        \item はじめに(introduction)
        \item 従来研究(related work)
        \item 提案手法(method)
        \item 実験結果、評価(result,evaluation)
        \item 議論(discussion)
        \item まとめ、今後の課題(conclusion)
    \end{enumerate}

    \section{参考文献の探し方}

    次のような論文データ収集サイトがあります。
    \begin{itemize}
        \item Google Scholar\\http://scholar.google.co.jp/
        \item CiteSeerX\\http://citeseerx.ist.psu.edu/
        \item DBLP\\http://www.informatik.uni-trier.de/~ley/db/
    \end{itemize}

    学会など公式データに基づいた論文検索のサイトです。
    \begin{itemize}
        \item ACM Digital Library\\http://dl.acm.org/
        \item IEEE Xplore\\http://ieeexplore.ieee.org/Xplore/guesthome.jsp
        \item CiNii\\http://ci.nii.ac.jp/
    \end{itemize}

    こういったところで検索をすると関連する論文を見つける可能性が高まります。
    有効に活用しましょう。

    %\bibliographystyle{acmsiggraph}
    \bibliographystyle{junsrt}
    %\nocite{*}
    \small
    \bibliography{references}
    %参考文献データの入ったファイル名をここに書
    %きます。

\end{document}
